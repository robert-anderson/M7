M7 ({\itshape Many-\/body Stochastic Expectation Value Estimation Networks}) is a stochastic quantum chemistry software package which primarily implements the F\+C\+I\+Q\+MC method \cite{doi:10.1063/1.3193710}. The purpose of this wiki is to provide a space in which M7 users and developers can learn from and contribute to a body of knowledge, advice, and best practices accumulated over the entire lifetime of this project.

The F\+C\+I\+Q\+MC method is founded on the assumption that stochastic application of the projector method recursion relation \[ \label{eq:master} \ketPsinext = (1-\tau\Hop)\ketPsin \] where the wavefunction is a linear superposition of Slater determinants \[ \label{eq:psidef} \ketPsi \equiv \sum_\bfi \Ci \ketDi \] can yield accurate eigenvalues and other properties for configuration-\/interation problems far beyond the scope of exact diagonalisation methods.

This delay in the onset of the \char`\"{}curse of dimensionality\char`\"{} faced by exponentially-\/scaling correlation methods is achieved by discretising the $\Ci$ coefficients as a population of {\itshape walkers}. Thus, at any given iteration of $\eqref{eq:master}$, the number of determinants with any walker occupation is usually a shrinking minority of the total dimension of the space (provided that a single particle basis conducive to this sparsity e.\+g. the canonical Hartree-\/\+Fock orbitals is chosen). By this approach, the memory scaling of the F\+CI problem can be brought within practicable limits, provided that an efficient system of data structures can be devised to store and update a sparse, parallelised representation of $\ketPsi$

Computational tractability is also aided by the walker discretisation, since the accumulation of walkers on a given determinant indicates the number of attempts the algorithm should make to convey new walkers from that source. The processes by which the $\Ci$ coefficients are updated are called {\itshape spawning} (off-\/diagonal elements), and {\itshape death} (off-\/diagonal elements) \[ \Cinext = \Cin - \tau \sum_{\bfj\neq\bfi} \Hij \Cjn - \tau(\Hii - S) \Cin \] 