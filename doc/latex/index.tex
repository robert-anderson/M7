Many-\/body Stochastic Expectation Value Estimation Networks (M7) is a stochastic electronic structure program written in C++ which primarily implements the Full Configuration Interaction Quantum Monte Carlo (F\+C\+I\+Q\+MC) method.

\section*{Installation}

To install M7, you will need\+:
\begin{DoxyItemize}
\item C++11 compiler which implements M\+PI
\item C\+Make installation at least as recent as version 3.\+10
\item L\+A\+P\+A\+CK implementation
\end{DoxyItemize}

Retrieve the program code from the V\+CS\+: 
\begin{DoxyCode}
1 git clone https://github.com/robert-anderson/M7.git
\end{DoxyCode}
 Then create a directory in which C\+Make can build its targets---typically under the repository root directory---and point C\+Make to the top level C\+Make\+Lists.\+txt file of the project. 
\begin{DoxyCode}
1 cd M7
2 mkdir build
3 cd build
4 cmake ..
5 make -j TARGET
\end{DoxyCode}
 Where T\+A\+R\+G\+ET is either debug, release or unittest. The binary will then appear at either src/debug, src/release, or test/unittest.

\section*{Usage}

The program takes all run time variables through the command line interface. The command line options available can be listed by executing the debug or release binaries with the -\/h or --help option.

Complex configurations can become cumbersome to manage on the command line directly, so particular run time configurations can easily be split over multiple lines in a shell script.

If the user chooses, the python/ directory contains a simple and convenient wrapper which encapsulates all options, and can handle the execution of distributed instances of M7. 